\documentclass[a4paper,11pt]{book}

\usepackage{cite}
\usepackage[utf8]{inputenc}
\usepackage[spanish]{babel}
\usepackage{fancyhdr}
\usepackage{hyperref}
\usepackage{graphicx}
\usepackage[a4paper]{geometry} %margenes
\usepackage[usenames,dvipsnames]{xcolor}
\usepackage{float}
\usepackage{anysize} 
\usepackage{listings}
\usepackage{paralist}
\hypersetup{
	pdfborder = {0 0 0}
}

\setcounter{secnumdepth}{3}
\setcounter{tocdepth}{2}
\usepackage{multicol}
\usepackage[T1]{fontenc}  
\usepackage{textcomp}  
\usepackage{lmodern} 

\usepackage{color}
\marginsize{3 cm}{3 cm}{2.5 cm}{2.5 cm}

\definecolor{javared}{rgb}{0.6,0,0} % for strings
\definecolor{javagreen}{rgb}{0.25,0.5,0.35} % comments
\definecolor{javapurple}{rgb}{0.5,0,0.35} % keywords
\definecolor{javadocblue}{rgb}{0.25,0.35,0.75} % javadoc
\definecolor{gray75}{gray}{.75}
 


\lstdefinestyle{consola}
   {basicstyle=\bf\ttfamily,
    backgroundcolor=\color{gray75},
   }
   
\lstset{language=Java,
basicstyle=\footnotesize,
keywordstyle=\color{javapurple}\bfseries,
stringstyle=\color{javared},
commentstyle=\color{javagreen},
morecomment=[s][\color{javadocblue}]{/**}{*/},
numbers=left,
numberstyle=\tiny\color{black},
stepnumber=1,
numbersep=10pt,
tabsize=2,
showspaces=false,
showstringspaces=false,
literate=%
   	{á}{{\'a}}1 {Á}{{\'A}}1 %
    {é}{{\'e}}1 {É}{{\'E}}1 %
    {í}{{\'i}}1 {Í}{{\'I}}1 %
    {ó}{{\'o}}1 {Ó}{{\'O}}1 %
    {ú}{{\'u}}1 {Ú}{{\'U}}1 %
    {ñ}{{\~n}}1 {Ñ}{{\~N}}1 %
    {¡}{{!`}}1 {¿}{{?`}}1 %
    {'}{\textquotesingle}1%
    {"}{\textquotedbl}1
}


% Limpia el encabezado en las páginas impares vacías
\makeatletter \def\cleardoublepage{
\clearpage\if@twoside \ifodd\c@page\else%
\hbox{}%
\thispagestyle{empty}% Aquí elimina el estilo del encabezado
\newpage%
\if@twocolumn\hbox{}\newpage\fi\fi\fi} \makeatother


\begin{document}



\pagestyle{empty}
\begin{titlepage}
\setlength{\unitlength}{1 cm} %Especificar unidad de trabajo
\thispagestyle{empty}
\begin{picture}(18,4)
\put(1,0){\includegraphics[width=0.145\textwidth]{./img/uco.png}}
\put(10,0){\includegraphics[width=0.2\textwidth]{./img/escudo.png}}
\end{picture}
\\
\\
\begin{center}
\textbf{{\Huge Universidad de Córdoba}\\[0.5cm]
{\LARGE Escuela Politécnica Superior de Córdoba}}\\[1.25cm]
{\Large \textbf{Sistemas en Tiempo Real}}\\
\Large Segundo Curso de Segundo Ciclo\\
\Large Ingeniería Informática\\
\Large Curso 2013/2014\\[1.5cm]
{\Huge \textbf{ \color{Brown}Memoria de prácticas}}\\[2cm]

{\Large Francisco Arjona López\\
Cristóbal Castro Villegas\\
José Antonio Espino Palomares\\
Antonio Osuna Caballero}\\[1.5cm]

\Large Córdoba, \today
\end{center}
\end{titlepage}

\cleardoublepage

%%\section*{Resumen}

En este trabajo se desea construir un reconocedor facial basado en el método de Análisis de Componentes Principales, PCA. Ésta es una técnica utilizada para reducir la dimensionalidad de un conjunto de datos. Se busca seleccionar aquellas características relevantes para poder discriminar de modo de no trabajar con la imagen original. De esta manera conseguimos acelerar los cálculos sin perder mucha información. De no utilizar esta técnica, el proceso de clasificación sería extremadamente lento ya que se haría sobre un espacio de una elevada dimensión.\\


\cleardoublepage
\pagestyle{plain}

\frontmatter % Introducción, índices ... 
\pagestyle{fancy}
\fancyhf{}
\fancyhead[LE]{\leftmark} % En las páginas impares, parte izquierda del encabezado, aparecerá el nombre de capítulo
\fancyhead[RO]{\leftmark} % En las páginas pares, parte derecha del encabezado, aparecerá el nombre de sección

\fancyfoot[C]{\thepage} % Números de página en las esquinas de los encabezados

\renewcommand{\chaptermark}[1]{\markboth{Capítulo \thechapter. #1}{}} % Formato para el capítulo: N. Nombre
\renewcommand{\sectionmark}[1]{\markright{\thesection. #1}} % Formato para la sección: N.M. Nombre

% Renombramos las figuras y las tablas
\renewcommand{\figurename}{Figura}
\renewcommand{\listfigurename}{Índice de figuras}
\renewcommand{\tablename}{Tabla}
\renewcommand{\listtablename}{Índice de tablas}
\tableofcontents

\addcontentsline{toc}{chapter}{Índice de figuras} %... y el de imágenes
\listoffigures
\addcontentsline{toc}{chapter}{Índice de tablas} %... el índice de tablas
\listoftables
\cleardoublepage

\mainmatter % Contenido en si ...


\chapter{Introducción}

\chapter{Práctica 1. Manejo básico de Lego Mindstorms NXT: Sensores y Actuadores}

\section{Objetivos}

Tras la realización de esta práctica el alumno debería ser capaz de:
\begin{itemize}
	\item Programar movimientos sincronizados del robot. 
	\item Conocer los parámetros de funcionamiento básico de los motores. 
	\item Utilización  de los sensores básicos. 
\end{itemize}

\section{Ejercicio A}
Calibrar la potencia relativa de los motores para realizar un movimiento lineal con un error menor a 1cm. de desvío por cada 1 m. de avance lineal. (Error menor al 1\%).

\section{Ejercicio B}

\par Programar una maniobra de aparcamiento del robot sin tener en cuenta sensores. Se muestra una descripción del entorno de aparcamiento (Figura~\ref{aparcamiento}). El robot se posicionará con las ruedas motrices rozando la línea lateral y con las ruedas justo por delante de la línea de inicio.

\begin{figure}[ht!]
 \centering
 \includegraphics[scale=0.7]{./img/recorrido1.png}
 \caption{Diagrama del aparcamiento.}
 \label{aparcamiento}
\end{figure}


\section{Ejercicio C}

\par Realizar un movimiento de avance del robot teniendo en cuenta los sensores de ultrasonidos y de pulsación, 
que estarán colocados en el frontal de avance del robot. El robot avanzará a plena potencia hasta que se encuentre 
a 1 m. de un obstáculo, que reducirá su potencia a la mitad. A 20 cm. de distancia del obstáculo, el robot reducirá 
su potencia de avance a un cuarto. El avance continuará hasta que el choque sea detectado por el sensor de pulsación, 
en cuyo caso, el robot retrocederá durante 1 seg. y girará 90º a la derecha, comenzando de nuevo el mismo procedimiento.




\chapter{Práctica 2. Manejo básico de Lego Mindstorms NXT: Tareas y Comunicaciones}

\chapter{Práctica 3. Manejo avanzado de Lego Mindstorms NXT}


\backmatter % Apéndices, bibliografia ...

\clearpage
%Añade la bibliografía al índice
\addcontentsline{toc}{chapter}{Bibliografía}
%La bibliografia aparece en el orden en que se cita, no alfabeticamente 
\bibliographystyle{unsrt}
\bibliography{bibliografia}


\end{document}